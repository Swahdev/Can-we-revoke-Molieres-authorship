\documentclass[]{report}


% Title Page

\title{Qui a écrit Molière ?}
\author{CHIKHANI Charles \and ZAPFACK MESSIAN Jasen Steve \\ \and Encadrant : M. Hervé Fournier\\ \\ Université Paris Cité}
\date{02 Juin 2023}

\begin{document}
	\maketitle
	
	\begin{abstract}
	\end{abstract}


	\maketitle
	\tableofcontents
	
	\newpage
\renewcommand{\thesection}{\arabic{section}}
	\section{Introduction}
	\subsection{Présentation du sujet : la remise en question de l'attribution des pièces de Molière à l'auteur lui-même}
\vspace{\baselineskip}
\hspace{0,5cm}Depuis plusieurs décennies, une question persiste dans le domaine
de la littérature : l'attribution des pièces de Molière à l'auteur lui-même
est-elle remise en question ? Cette controverse a été initiée par Pierre Louÿs,
un romancier du XXe siècle, qui a suggéré que Pierre Corneille aurait pu être
l'auteur véritable des pièces de Molière. Cependant, cette théorie repose sur
des fondements fragiles et ne bénéficie d'aucune preuve concrète. Malgré cela,
la rumeur persiste, alimentée par des éléments tels que l'éclosion tardive de
Molière en tant qu'auteur, son prétendu manque d'éducation et de culture, ainsi
que l'absence de preuves manuscrites permettant de réfuter directement cette
hypothèse.


\vspace{\baselineskip}
\subsection{Description de l'expérience que nous avons menée}
\vspace{\baselineskip}
\hspace{0,5cm}Dans le cadre de cette étude, nous avons mené une expérience
visant à déterminer la paternité des textes de Molière. L'objectif principal
était de développer une méthodologie pour identifier les caractéristiques
distinctives du style d'écriture de Molière.Nous comparons ensuite les textes de
Moliere et ces de Corneille.

\subsection{Choix des outils d'analyse textuelle et statistique utilisés}
\vspace{\baselineskip}
\hspace{0,5cm} Pour mener a bien notre expérience, nous avons opté pour la
bibliothèque  \textbf{NLTK} (\textit{Natural Language Toolkit}) de Python,très
Utiliser dans la discipline du \textbf{NLP} (\textit{Natural Language
Processing}).Ensuite nous utilisons aussi l'agorithme open-source de Facebook,
\textbf{fasttext} qui nous permets de vectoriser chacun de notre text

pour notre approche Machine learning, nous avons utilisé la librairie
\textit{scikit-learn} de python qui offre des fonctionnalités avancées pour
l'analyse de texte et la classification telle que algorithme de clusterisation
hierachique et le K-means 
\subsection{Nos etapes } 

\begin{itemize}
\item Notre corpus est constituer des Oeuvres de Corneile en format pdf, et
celle de Moliere  en format xml, que nous avons prises depuis le depot git des
chercheurs 
\item le prétraitement de notre corpus se base sur la caractéristique des
affixes dans notre textes. Cette étape nous fournira des fichier texte des
œuvres de molières et de corneille contenant la listes de tout les mots sans
suffixes, la complexité en temps globale de cette étape est de O(n\(^2)\)
\item la vectorisation des fichiers txt.
\item Nous avons appliqué nos algorithmes de clustering, qui comprennent le
k-means utilisant la distance euclidienne et la métrique du cosinus, ainsi que
l'agglomerative Clustering utilisant la distance de Jaccard. Les mesures
d'agrégation des clusters que nous avons utilisées sont ``complete'' et
``average''. En ce qui concerne la complexité de nos algorithmes, elle dépend de
plusieurs facteurs tels que la taille des données et le nombre de clusters. 
\item  on a la les dendrogramme correspondant a chaque métrique avec leurs
critère d'union
\end{itemize}

\subsection{Observation}
Comme nous l'avons constaté lors de l'application de l'algorithme hiérarchique
sur des données textuelles, l'utilisation de la distance euclidienne, telle
qu'utilisée par D. Labbé, ou la métrique du cosinus, ont montré une moindre
fiabilité et conduisent à des conclusions différentes de celles obtenues en
utilisant une métrique de Jaccard, qui se base davantage sur une approche
ensembliste 

\subsection{Nos différents résultas}
Pendant nos expériences pratiques, nous avons eu l'occasion de tester
différentes méthodes d'algorithme de regroupement (clusterisation) de diverses
manières, en nous appuyant sur différentes caractéristiques. Par exemple :

\begin{itemize}
\item Après avoir analysé les digrammes et les trigrammes, nous avons comparé
les œuvres de Molière et de Corneille. Les résultats se présentent sous la forme
d'un fichier texte répertoriant toutes les paires de mots différents fréquemment
associés ainsi que les groupes de trois mots (trigrammes) qui apparaissent
ensemble avec leur fréquence respective.
	
Une observation intéressante est que le digramme (``peut'', ``être'') est plus
fréquent, avec 235 occurrences, dans les œuvres de Molière, tandis que chez
Corneille, c'est le digramme (``après'', ``avoir'') qui revient le plus souvent,
avec 117 occurrences.
	
En ce qui concerne les trigrammes, le triplet de mots le plus fréquent chez
Molière est (``monsieur'', ``oui'', ``monsieur''), tandis que chez Corneille,
c'est (``plus'', ``puis'', ``faire'').

	En outre, grâce à l'analyse des nuages de mots, nous constatons une
	utilisation quasi similaire du mot ``plus'`. Cette utilisation fréquente du mot
	``plus'' par nos différents auteurs peut rendre difficile l'attribution précise de
	l'auteur en se basant uniquement sur la distance euclidienne inter-textuelle,
	comme celle proposée par D. Labbé. Cela souligne l'importance d'utiliser des
	métriques et des méthodes d'analyse supplémentaires pour une identification
	plus précise de l'auteur

\end{itemize}
\subsection{interpretation dendrogramme}

Lors de l'interprétation des résultats sur les dendrogrammes, nous constatons
que certaines œuvres ne sont pas bien positionnées lorsque nous utilisons la
distance euclidienne et la métrique du cosinus. En revanche, avec la métrique de
Jaccard, nous observons des regroupements plus compacts, que ce soit pour les
œuvres de Molière ou de Corneille. Cela suggère que la métrique de Jaccard est
plus efficace pour capturer les similitudes et les différences entre les textes,
grâce a son approche ensembliste qui le différencie des autre métrique. Ce qui
s'appui dans nos étude, car nous nous appuyons principalement sur la
méthodologie développée par Caffiero et Camps, car nous considérons que leur
approche et leurs choix de caractéristiques et de métriques sont plus pertinents.
Nous reconnaissons que leur travail fournit une base solide et cohérente pour
notre propre recherche, et nous sommes en accord avec leur méthodologie.

\vspace{\baselineskip} 
\hspace{0,5cm} D'après nos différentes expériences, nous
pouvons conclure que le style d'écriture de Molière présente une unicité
remarquable. Les caractéristiques spécifiques que nous avons observées dans ses
œuvres, telles que l'utilisation fréquente de certains bigrammes et trigrammes,
ainsi que des motifs récurrents et la fréquence des mots lemmatisés, nous ont
permis d'identifier et de distinguer son style d'écriture par rapport à celui
des autres auteurs. Ces résultats renforcent l'idée que Molière possède un style
d'écriture unique et reconnaissable.

\end{document}