\vspace{\baselineskip}
\hspace{0,5cm}La sélection d'une caractéristique fiable et informative est une étape cruciale
dans la réalisation d'une étude statistique. En effet, les caractéristiques ont
pour but d'améliorer la fiabilité des des analyses. La sélection d'une
caractéristique s'effectue en fonction de la taille du corpus, le niveau de
confiance et la marge d'erreur potentiel. 

\hspace{0,5cm} Grâce à cette formule, la taille minimale de l'échantillon notée \textit{n} a
été calculée en utilisant la formule :
\[n=p(1-p)(z/e)^2\]
où \textit{p} représente la probabilité moyenne de la caractéristique dans notre
corpus, \textit{z} le niveau de confiance et \textit{e} la marge d'erreur de
l'estimation de probabilité.
\\Dans l'étude, \textit{z} a été fixé de manière à obtenir un niveau de confiance
qui est supérieur à 90 et \textit{e} = 2\textit{s} où \textit{s} est
l'écart-type de la caractéristique dans le corpus.

\hspace{0,5cm}Toutes caractéristiques sélectionnées doivent suivre une
distribution gaussienne.
