\vspace{\baselineskip}
\hspace{0,5cm} Dans ce rapport, nous examinerons deux hypothèses qui remettent
en question la
paternité des œuvres de Molière. La première hypothèse suggère que Molière
aurait fourni des brouillons à Pierre Corneille, qui aurait ensuite versifié les
pièces, peut-être avec l'aide de son frère. Selon cette hypothèse, Molière
aurait créé les intrigues, mais la versification aurait été réalisée par Pierre
Corneille, sans recevoir un crédit explicite. La deuxième hypothèse soutient que
Molière n'aurait ni écrit les intrigues ni les vers de ses pièces, et qu'il
n'aurait été qu'un nom célèbre utilisé pour promouvoir les pièces et dissimuler
le véritable auteur.

\hspace{0,5 cm} Pour résoudre cette controverse, différentes méthodes ont été
utilisées, chacune
avec ses propres avantages et limitations. L'objectif de cette étude est
d'évaluer ces approches et de fournir une analyse critique des résultats
obtenus.
\\Pour ce faire, dans la section suivante, nous présenterons en détail l'article de Florian
Cafiero et Jean-Baptiste Camps, qui examine cette question en utilisant une
analyse textuelle et statistique. Nous aborderons les différentes perspectives
et critiques soulevées par d'autres chercheurs, ainsi que les différentes
approches méthodologiques qui ont été utilisées pour résoudre le problème de
l'attribution des pièces de Molière.


Le rapport qui suit explorera en profondeur ces différentes approches et tentera
de faire la lumière sur cette controverse persistante, fournissant ainsi une
contribution significative à notre compréhension de l'œuvre de Molière et de son
véritable auteur.
