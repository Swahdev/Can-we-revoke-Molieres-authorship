\vspace{\baselineskip}
\hspace{0,5cm}Sur chacun de ces corpus, les chercheurs ont appliqué les caractéristiques
suivantes :

\hspace{1cm}- \textbf{Lexicon} : un lexique désigne l'ensemble des mots et des unités lexicales d'une
langue, incluant leurs sens, leurs formes grammaticales et leurs relations. Par
exemple, "maison", "chien", "arbre".

\hspace{1cm}- \textbf{Rhyme Lexicon} : fait référence à un lexique spécifique aux rimes. Il s'agit
d'une liste qui répertorie les mots et les expressions en fonction de leurs
sonorités et de leurs similarités phonétiques, notamment la dernière syllabe ou
les sons finaux des mots. Par exemple, "rat", "chat", "chapeau", "bateau",
"plateau" - tous les mots se terminent par le son "-o" ou "-au".

\hspace{1cm}- \textbf{Affixes} : les affixes désignent les éléments qui peuvent être ajoutés aux mots
pour en modifier le sens ou la fonction. Ils comprennent les préfixes (ajoutés
au début du mot), les suffixes (ajoutés à la fin du mot) et les infixes (ajoutés
à l'intérieur du mot).

\hspace{1cm}- \textbf{Morphosyntactic sequences} : il s'agit de l'analyse des séquences
morphosyntaxiques, c'est-à-dire l'étude des combinaisons de morphèmes et de
structures grammaticales dans une phrase.

\hspace{1cm}- \textbf{Mots fonctionnels / mots-outils} : ce sont des mots grammaticaux
qui ont principalement un rôle syntaxique ou grammatical dans une phrase, plutôt
qu'un sens lexical spécifique. Les mots fonctionnels comprennent souvent les
prépositions, les conjonctions, les pronoms, les déterminants, les adverbes de
liaison et les particules grammaticales. Exemples : "de", "à", "dans", "sur",
"sous", etc.