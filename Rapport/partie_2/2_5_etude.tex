\vspace{\baselineskip}
\hspace{0,5cm}Le choix de la mesure de distance et du critère de liaison (liaison complète,
simple ou moyenne) détermine la manière dont la similarité entre les clusters
est évaluée lors du processus de fusion.

\hspace{0,5cm}La métrique de distance utilisée dans cet algorithme permet de quantifier la
similarité ou la dissimilarité entre les points de données ou les clusters. Elle
détermine comment l'algorithme mesure la distance ou la dissimilarité entre les
observations afin de former des clusters. Parmi les exemples de métriques
couramment utilisées, on retrouve la \textit{distance euclidienne}, \textit{la
distance de Manhattan} et \textit{la distance cosinus}.

\hspace{0,5cm}Dans l'étude, les chercheurs ont utilisé la distance de \textit{Burrow's delta}
et le \textit{min-max}. Nous allons étudier la distance de Burrow.
Cette distance calcule la distance de Manhattan entre les \textit{z-scores} des
fréquences de ces caractéristiques dans les textes de deux auteurs.  Le
\textit{z-scores} est une notion statistique qui quantifie le nombre
d'écart-types par lequel une observation ou un point de données s'éloigne de la
moyenne d'une distribution.  La distance de Borrow mesure donc la dissimilarité
entre le style d'écriture des deux textes comparés, en prenant en compte les
différences dans les fréquences normalisées de ces caractéristiques.
\\La formule de calcule de la distance de Burrow :
\[ \delta(A,B) = \sum_{i=1}^n \left| \frac{{(A_i - B_i)}}{{\sigma_i}} \right| \]
avec $A_i$ et $B_i$ des fréquences de mots dans le texte et $\sigma_i$ la
variance de l'utilisation du mot.

\vspace{\baselineskip}
\hspace{0,5cm}La métrique d'union (\textit{linkage} en anglais) est une méthode utilisée dans
le \textbf{regroupement hiérarchique} pour calculer la distance entre les
clusters lors du processus de fusion. Elle est utilisée pour déterminer comment
les clusters sont regroupés pour former des clusters plus grands. Différentes
méthodes de linkage, telles que le ward linkage, le single linkage, etc.,
peuvent être utilisées.
\\La formule de calcul de distance entre les clusters utilisée dans la recherche
se base sur la métrique d'union. Prenons l'exemple de deux clusters
\textit{$C_1$} et \textit{$C_2$}, avec leurs centroides respectifs
\textit{$G_1$} et \textit{$G_2$}, et les nombres d'individus dans les clusters
\textit{$n_1$} et \textit{$n_2$}. La distance d entre les clusters, à minimiser,
est définie par l'équation suivante :
\[d^2( C_1,C_2) = ((n_1 * n_2) / (n_1 + n_2)) * d^2(G_1,G_2)\]

L'objectif du regroupement hiérarchique est de minimiser la distance entre les
clusters lors de la fusion, ce qui peut être réalisé en choisissant la méthode
de linkage appropriée et en ajustant les paramètres en conséquence.