\documentclass[]{report}


% Title Page

\title{Qui a écrit Molière ?}
\author{CHIKHANI Charles \and ZAPFACK MESSIAN Jasen Steve \\ \and Encadrant : M. Hervé Fournier\\ \\ Université Paris Cité}
\date{02 Juin 2023}

\begin{document}
	\maketitle
	
	\begin{abstract}
	\end{abstract}


	\maketitle
	\tableofcontents
	
	\newpage
\renewcommand{\thesection}{\arabic{section}}
	\section{Introduction}
	\subsection{Présentation du sujet : la remise en question de l'attribution des pièces de Molière à l'auteur lui-même}
\vspace{\baselineskip}
\hspace{0,5cm}Depuis plusieurs décennies, une question persiste dans le domaine
de la littérature : l'attribution des pièces de Molière à l'auteur lui-même
est-elle remise en question ? Cette controverse a été initiée par Pierre Louÿs,
un romancier du XXe siècle, qui a suggéré que Pierre Corneille aurait pu être
l'auteur véritable des pièces de Molière. Cependant, cette théorie repose sur
des fondements fragiles et ne bénéficie d'aucune preuve concrète. Malgré cela,
la rumeur persiste, alimentée par des éléments tels que l'éclosion tardive de
Molière en tant qu'auteur, son prétendu manque d'éducation et de culture, ainsi
que l'absence de preuves manuscrites permettant de réfuter directement cette
hypothèse.

\subsection{Importance de l'analyse textuelle et statistique dans la résolution de cette question}
\vspace{\baselineskip}
\hspace{0,5cm}Au début des années 2000, Cyril et Dominique Labbé, deux
chercheurs, ont avancé l'idée selon laquelle Corneille aurait écrit pour
Molière. Leur méthode consiste à mesurer une "distance inter-textuelle" qui
évalue la différence de lexique entre les textes des deux auteurs. Si cette
distance ne dépasse pas un certain seuil, les deux pièces sont considérées comme
écrites par le même auteur. Ils se basent également sur le fait que de nombreux
dramaturges de l'époque signaient leurs œuvres sous le nom de "comédien poète",
ce qui permettait aux véritables auteurs de rester anonymes tout en bénéficiant
de la promotion et de la représentation de leurs pièces par les acteurs.


\hspace{0,5cm}Cependant, cette méthodologie a été contestée par d'autres
chercheurs. Certains ont souligné que l'implémentation de la méthode de Cyril et
Dominique Labbé pourrait "lisser artificiellement les différences entre les
auteurs", en utilisant une distance euclidienne qui accorde trop de poids aux
lemmes fréquents, réduisant ainsi la disparité entre les fréquences observées de
différentes formes.  \\D'autres approches ont été proposées pour résoudre le
problème de l'attribution des comédies de Molière. Certaines méthodes utilisent
une analyse textuelle et statistique pour comparer les styles d'écriture, tandis
que d'autres adoptent des approches plus qualitatives en examinant les
intrigues, la versification et les sujets choisis dans les pièces.

\subsection{Hypothèses et problématique}
\vspace{\baselineskip}
\hspace{0,5cm} Dans ce rapport, nous examinerons deux hypothèses qui remettent
en question la paternité des œuvres de Molière.
\begin{itemize}
\item 	La première hypothèse suggère que Molière aurait fourni des brouillons à
Pierre Corneille, qui aurait ensuite versifié les pièces, peut-être avec l'aide
de son frère. Selon cette hypothèse, Molière aurait créé les intrigues, mais la
versification aurait été réalisée par Pierre Corneille, sans recevoir un crédit
explicite. La deuxième hypothèse soutient que Molière n'aurait ni écrit les
intrigues ni les vers de ses pièces, et qu'il n'aurait été qu'un nom célèbre
utilisé pour promouvoir les pièces et dissimuler le véritable auteur.
	
\item La deuxième hypothèse, suggère que Molière n'aurait pas écrit ni les
intrigues ni les vers de ses pièces, et qu'il n'aurait été qu'un nom célèbre
utilisé pour aider à promouvoir la pièce, pour satisfaire l'ego de l'acteur
principal/metteur en scène et pour dissimuler le nom de l'auteur réel. Selon
cette hypothèse, les sujets choisis dans les pièces de Molière, comme les
Précieuses Ridicules, auraient été plus proches des intérêts habituels de P. (ou
T.) Corneille et ne refléteraient aucune influence de Molière. Si cela était
vrai, tous les indicateurs devraient montrer que le vocabulaire et le style de
Molière n'existent pas, et les pièces de Molière devraient être confondues avec
celles de P. Corneille selon chacun des six critères évalués dans cette étude.
\end{itemize}
 

\section{Compréhension de l'article de Florian Cafiero et Jean-Baptiste Camps}
\hspace{0,5 cm} Le but de nos chercheurs sera de réfuter ces 2 hypothèse, nous
allons plus en parler de la méthodologie et la procédure pour le faire

\subsection{Méthodologie}
\vspace{\baselineskip}
\hspace{0,5cm} cela sera constituer de 3 sets de d'oeuvres que nous appellerons
corpus.(def. un corpus désigne une collection importante et structurée de textes
ou de documents utilisée pour l'analyse linguistique) et cela peut comprend
généralement variété de textes telle que des livres, des articles et etc.  
\begin{itemize}
\item   le première le corpus exploratoire est constitué d'un large échantillon
de comédies en vers. Cet échantillon comprend des pièces d'au moins 5000 mots,
pour les auteurs ayant écrit au moins trois comédies. Il inclut des pièces de
théâtre de 12 auteurs.
	
\item   le deuxième corpus  le corpus final est construit  Pour obtenir un
résultat plus lisible et moins biaisé, ils vont se concentrée sur les
sous-genre. Afin d'éviter les biais liés aux sous-genres, ils vont exclure les
comédies héroïques et les courtes farces comiques. Pour éliminer le bruit ajouté
par de nombreux phénomènes (co écriture, plagiat, attribution incertaine, etc.)
sans rapport avec les hypothèses présenter plus tôt, ils choisissent de
concentrer uniquement sur cinq auteurs majeurs de l'époque. Ce corpus final
comprend 37 pièces de T. et P. Corneille, Molière, Rotrou et Scarron
	
\item le troisième corpus qui les sert de test pour vérifier la précision de
leur approche et consiste de comédie en vers écrit après la mort de P.Corneille
et Molière	
\end{itemize}

\subsection{les caractéristique d'étude de texte (studies features)}
sur chacun de ces corpus , nos chercheurs on appliquer les caractéristiques qui
suit
\begin{itemize}
\item Lexicon: un lexicon désigne l'ensemble des mots et des unités lexicales
d'une langue, ainsi que leurs sens, leurs formes grammaticales et leurs
relations. Des exemples sont Maison, Chien, Arbre.
\item Rhyme Lexicon: fait référence à un lexique spécifique aux rimes.Il s'agit
d'une liste qui répertorie les mots et les expressions en fonction de leurs
sonorités et de leurs similarités phonétiques, en particulier en ce qui concerne
la dernière syllabe ou les sons finaux des mots. exemple  rat, chat, chapeau,
bateau, plateau 	tous les mots se terminent par le son "-o" ou "-au".
\item Word forms
	
\item Affixes
	
\item Morphosyntactic sequences 
	
\item Mot Fontionelle/ mot-outils (function words): sont des mots grammaticaux
qui ont principalement un rôle syntaxique ou grammatical dans une phrase plutôt
qu'un sens lexical spécifique.Les mots fonctionnels sont souvent des
prépositions, des conjonctions, des pronoms, des déterminants, des adverbes de
liaison et des particules grammaticales. Comme les prépositions : de, à, dans,
sur, sous etc.
\end{itemize}
\subsection{Choisir la caractéristique }
\vspace{\baselineskip}
\hspace{0,5cm} En Générale, La sélection des caractéristiques les plus fiables
et informatives pour l'analyse stylistique de texte est une question qui a fait
l'objet de nombreuses contributions.  Afin d'augmenter la fiabilité des
analyses, dans un corpus contenant des textes de longueurs variables, ils ont
décidé de sélectionner des caractéristiques  avec une approche  statisticienne
en fonction du niveau de confiance et de la marge d'erreur que nous pouvions
obtenir même pour le plus petit échantillon disponible dans notre corpus.

La taille minimale de l'échantillon ,n, a été calculée en utilisant la formule
suivante où p est la probabilité moyenne de la caractéristique dans notre
corpus, utilisée comme estimation de la probabilité de la population pie, z est
le niveau de confiance et e est la marge d'erreur de l'estimation de
probabilité.  Nous avons fixé z de manière à obtenir un niveau de confiance
supérieur à 90 et e = 2s, où s est l'écart-type de la caractéristique dans le
corpus.
\[n=p(1-p)(z/e)^2\]

mais pour cela les caractéristique doivent suivre une distributions gaussiennes
\subsection{ Algorithme de Clusterization}

Dans une approche d'analyse statisticien dans l’attribution d’auteur grâce au
machine Learning , Algorithme de clusterization hiérarchique est appliqué a
chacun des corpus cité plus haut.  cette algo est un type spécifique
d'algorithme de regroupement utilisé en apprentissage automatique et en analyse
de données. Il s'agit d'une approche ascendante où chaque point de données est
considéré initialement comme un cluster séparé, puis fusionné de manière
itérative en fonction de leur similarité

pour un rappel un cluster dans le contexte de l'analyse de regroupement, désigne
un groupe de points de données partageant des similitudes ou présentant des
schémas lorsqu'ils sont comparés à d'autres points de données. Ces points de
données sont regroupés en fonction de certains critères, tels que la proximité
dans l'espace des caractéristiques ou la similarité dans les valeurs des
attributs. Les clusters sont formés en fonction des mesures de similarité ou de
dissimilarité utilisées dans l'algorithme de regroupement.

les mesures de similarité et de dissimilarité sont calculer grâce a des métrique

\subsection{Métrique de algorithme }
Le choix de la mesure de distance et du critère de liaison (par exemple, liaison
complète, simple ou moyenne) détermine la manière dont la similarité entre les
clusters est mesurée lors du processus de fusion.
\begin{itemize}
\item la métrique  de distance entre point utiliser dans cette algo la
similarité ou la dissimilarité entre les points de données ou les clusters. Elle
détermine comment l'algorithme quantifie la distance ou la dissimilarité entre
les observations afin de former des clusters.  exemple de métrique couramment
utilisées , la distance euclidienne, la distance de Manhattan, métrique de
cosinus.
	
Nos chercheurs ont quand a eux utilisés la distance de Burrow's delta et le
min-max. Nous détaillerons la distance de Burrow.
	
cette distance calcule la distance de Manhattan entre les scores z des
fréquences de ces caractéristiques dans les textes de deux auteurs. Elle mesure
la dissimilarité entre leurs styles d'écriture en prenant en compte les
différences dans les fréquences normalisées de ces caractéristiques.
	  
rappel le Z-score est une notion de statistique qui quantifie le nombre
d'écart-types une observation ou un point de données est éloigné de la moyenne
d'une distribution.
	  
la formule de calcule de la distance de Burrow :
\[ delta(A,B)= \sum_{i=1}^n abs((A_i - B_i)/ \sigma i ) \] où les Ai et Bi sont
des fréquence de mots dans le texte.  Le sigma i est la variance de utilisation
du mot
	  
\item la métrique d'union(linkage) une méthode spécifique utilisée dans le
regroupement hiérarchique pour déterminer la distance entre les clusters lors du
processus de fusion.Par exemple le ward linkage , single linkage etc 
	
le calcul de distance se base sur cette formule,  prenons l'exemple le cluster
C1 et C2,  G1 et G2 leurs centroides respectifs,  n1 et n2 le nombre d'individus
dans les clusters respectifs.  La distance d entre les clusters, à minimiser,
est définie par l'équation suivante :
	
\[d^2( C1,C2) = n1 * n2 / n1 + n2 . d^2(G1,G2)\]
	 
\end{itemize}
\subsection{ Dendrogramme }
Après l'application de l'Algorithme de clusterization sur chacun des corpus. On
a comme résultat, un dendrogramme pour chacun des caractéristique   

% TODO: \usepackage{graphicx} required
\begin{center}
% \includegraphics{dendrogramme}
\end{center}
 \vspace{\baselineskip}
\hspace{0,5cm}  de cette interprétation , on en ressort avec une réfutation des
hypothèse énonce plus haut car comme indiquer sur la photo on peut voir une
distinction du cluster de Molière comparer au autre auteur.
\section{Notre expérience}

\vspace{\baselineskip}
\subsection{Description de l'expérience que nous avons menée}
\vspace{\baselineskip}
\hspace{0,5cm}Dans le cadre de cette étude, nous avons mené une expérience
visant à déterminer la paternité des textes de Molière. L'objectif principal
était de développer une méthodologie pour identifier les caractéristiques
distinctives du style d'écriture de Molière.Nous comparons ensuite les textes de
Moliere et ces de Corneille.

\subsection{Choix des outils d'analyse textuelle et statistique utilisés}
\vspace{\baselineskip}
\hspace{0,5cm} Pour mener a bien notre expérience, nous avons opté pour la
bibliothèque  \textbf{NLTK} (\textit{Natural Language Toolkit}) de Python,très
Utiliser dans la discipline du \textbf{NLP} (\textit{Natural Language
Processing}).Ensuite nous utilisons aussi l'agorithme open-source de Facebook,
\textbf{fasttext} qui nous permets de vectoriser chacun de notre text

pour notre approche Machine learning, nous avons utilisé la librairie
\textit{scikit-learn} de python qui offre des fonctionnalités avancées pour
l'analyse de texte et la classification telle que algorithme de clusterisation
hierachique et le K-means 
\subsection{Nos etapes } 

\begin{itemize}
\item Notre corpus est constituer des Oeuvres de Corneile en format pdf, et
celle de Moliere  en format xml, que nous avons prises depuis le depot git des
chercheurs 
\item le prétraitement de notre corpus se base sur la caractéristique des
affixes dans notre textes. Cette étape nous fournira des fichier texte des
œuvres de molières et de corneille contenant la listes de tout les mots sans
suffixes, la complexité en temps globale de cette étape est de O(n\(^2)\)
\item la vectorisation des fichiers txt.
\item Nous avons appliqué nos algorithmes de clustering, qui comprennent le
k-means utilisant la distance euclidienne et la métrique du cosinus, ainsi que
l'agglomerative Clustering utilisant la distance de Jaccard. Les mesures
d'agrégation des clusters que nous avons utilisées sont ``complete'' et
``average''. En ce qui concerne la complexité de nos algorithmes, elle dépend de
plusieurs facteurs tels que la taille des données et le nombre de clusters. 
\item  on a la les dendrogramme correspondant a chaque métrique avec leurs
critère d'union
\end{itemize}

\subsection{Observation}
Comme nous l'avons constaté lors de l'application de l'algorithme hiérarchique
sur des données textuelles, l'utilisation de la distance euclidienne, telle
qu'utilisée par D. Labbé, ou la métrique du cosinus, ont montré une moindre
fiabilité et conduisent à des conclusions différentes de celles obtenues en
utilisant une métrique de Jaccard, qui se base davantage sur une approche
ensembliste 

\subsection{Nos différents résultas}
Pendant nos expériences pratiques, nous avons eu l'occasion de tester
différentes méthodes d'algorithme de regroupement (clusterisation) de diverses
manières, en nous appuyant sur différentes caractéristiques. Par exemple :

\begin{itemize}
\item Après avoir analysé les digrammes et les trigrammes, nous avons comparé
les œuvres de Molière et de Corneille. Les résultats se présentent sous la forme
d'un fichier texte répertoriant toutes les paires de mots différents fréquemment
associés ainsi que les groupes de trois mots (trigrammes) qui apparaissent
ensemble avec leur fréquence respective.
	
Une observation intéressante est que le digramme (``peut'', ``être'') est plus
fréquent, avec 235 occurrences, dans les œuvres de Molière, tandis que chez
Corneille, c'est le digramme (``après'', ``avoir'') qui revient le plus souvent,
avec 117 occurrences.
	
En ce qui concerne les trigrammes, le triplet de mots le plus fréquent chez
Molière est (``monsieur'', ``oui'', ``monsieur''), tandis que chez Corneille,
c'est (``plus'', ``puis'', ``faire'').
	
\end{itemize}
\section{Conclusion}
\vspace{\baselineskip}
\hspace{0,5cm} À partir de nos différentes expériences, nous pouvons conclure
que le style d'écriture de Molière présente une unicité distincte. Les
caractéristiques spécifiques que nous avons observées dans ses œuvres, telles
que l'utilisation fréquente de certains digrammes et trigrammes, ainsi que des
motifs récurrents, ont contribué à identifier et distinguer son style d'écriture
des autres auteurs. Ces résultats renforcent l'idée que Molière possède un style
d'écriture unique et reconnaissable.
\end{document}          
