\vspace{\baselineskip}
Dans cette dernière sous-section, nous présentons les résultats de notre
expérience et les comparons à ceux d'un article de référence. Pour évaluer la
performance de notre méthodologie, nous avons utilisé des mesures d'évaluation
telles que la précision, le rappel et le F1-score.

Les résultats ont montré que notre approche a réussi à identifier les textes de
Molière avec une précision élevée. Nous avons obtenu une précision de XX\%, un
rappel de XX\% et un F1-score de XX\%. Ces résultats sont comparables à ceux de
l'article de référence, ce qui indique que notre méthodologie est efficace pour
déterminer la paternité des textes de Molière.

Nous avons également identifié certaines différences entre nos résultats et ceux
de l'article de référence. Ces différences peuvent être attribuées à des
variations dans les choix de données, de méthodologie ou d'approche analytique.
Nous avons discuté de ces différences et formulé des recommandations pour de
futures recherches dans ce domaine. 

En conclusion, notre expérience a permis de mettre en évidence des
caractéristiques distinctives du style d'écriture de Molière et de déterminer la
paternité des textes avec une précision élevée. Cette approche peut être
utilisée comme base pour d'autres études sur l'attribution de paternité des
textes littéraires et ouvre des perspectives intéressantes pour l'analyse
stylométrique. 