\vspace{\baselineskip}
\hspace{0,5cm}Pour mener notre analyse, nous avons choisi plusieurs outils d'analyse textuelle
et statistique. Nous avons opté pour la bibliothèque \textbf{NLTK} (\textit{Natural Language
Toolkit}) de Python pour ses fonctionnalités de prétraitement de texte, telles
que la suppression des \textit{stopwords}, la normalisation des mots et la lemmatisation.
Cette bibliothèque nous a permis de nettoyer les données textuelles et de
réduire les informations redondantes ou inutiles.
\\Nous utilisons également le modèle pré-entrainé \textbf{FastText} de Facebook
pour la vectorisation des mots.
\\En ce qui concerne l'analyse statistique, nous avons utilisé des techniques
telles que l'analyse de fréquence des mots, l'analyse des \textit{n-grammes} et l'analyse
de similarité. Pour ces tâches, nous avons utilisé des bibliothèques \textit{Python}
telles que \textit{scikit-learn}, qui offre des fonctionnalités avancées pour l'analyse
de texte et la classification et la classification telle que la clusterisation
hiérarchique et les K-means.
