\vspace{\baselineskip}
\hspace{0,5cm}La méthodologie mise en œuvre repose sur un ensemble d'étapes soigneusement
orchestrées :
\\La collecte des données a été réalisée avec rigueur et minutie. Une vaste
compilation d'œuvres signées Molière ainsi que celles d'autres auteurs
contemporains a été rassemblée afin de constituer le corpus de textes sur lequel
s'appuie notre étude. 
\\Le prétraitement des données a été effectué avec une approche méthodologique
précise. Des techniques de prétraitement textuel sophistiquées ont été
appliquées pour purifier les données, éliminant ainsi les éléments superflus et
les interférences indésirables. Parmi ces techniques, on compte la suppression
des stopwords, la normalisation des mots et la lemmatisation. 
\\L'analyse des caractéristiques textuelles a été entreprise avec une attention
particulière portée à chaque auteur. Par le biais de techniques d'analyse
textuelle avancées, notamment l'analyse de fréquence des mots, l'exploration des
n-grammes et l'évaluation de la similarité, nous avons extrait et étudié en
profondeur les traits caractéristiques propres au style d'écriture de chacun. 
\\La classification des textes a constitué une étape cruciale de notre démarche.
Nous avons employé des méthodes de classification établies, telles que les
arbres de décision et les algorithmes de classification bayésienne, pour
attribuer à chaque segment de texte son auteur d'origine, qu'il s'agisse de
Molière ou d'un autre écrivain de référence. 