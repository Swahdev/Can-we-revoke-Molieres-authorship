\vspace{\baselineskip}
\hspace{0,5cm}Pour notre expérience, nous avons pris une collection d'œuvres de Molière,
ainsi que des œuvres de Corneille. Ces données ont été
recueillies à partir de sources disponibles en ligne. Pour Molière nous avons
téléchargé des fichiers XML du GitHub de Mr Cafiero de 38 pièces. Pour Corneille
nous avons téléchargé des fichiers PDF du site Wikisource pour un corpus composé
de 34 pièces.

\vspace{\baselineskip}

\hspace{0,5cm}Le pré-traitement des données utilisées pour l'analyse de texte,
c'est à dire, les bigrammes, les trigrammes et les mots les plus fréquents, a été
réalisé en convertissant les fichiers PDF et XML en fichiers texte. Chaque œuvre
est donc convertit en une liste de mots en supprimant les \textit{mots vides} et
la ponctuation. Les mots vides sont des mots qui n'ont pas de signification et
ne sont pas pertinents pour l'analyse de texte.

\hspace{0,5cm}Pour l'analyse de similarité, le pré-traitement des données a été
légèrement different. Comme nous avons fait le choix d'utiliser la caractéristique
\textit{Affixe}, nous avons, grâce à une fonction de la bibliothèque NLTK, pu
supprimer les suffixes des mots. Les \textit{mots vides} et la ponctuation ont
également été retirés.
