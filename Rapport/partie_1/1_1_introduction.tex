\vspace{\baselineskip}
\hspace{0,5cm}Depuis plusieurs décennies, une question persiste dans le domaine de la
littérature : l'attribution des pièces de Molière à l'auteur lui-même est-elle
remise en question ? Cette controverse a été initiée par Pierre Louÿs, un
romancier du XXe siècle, qui a suggéré que Pierre Corneille aurait pu être
l'auteur véritable des pièces de Molière. Cependant, cette théorie repose sur
des fondements fragiles et ne bénéficie d'aucune preuve concrète. Malgré cela,
la rumeur persiste, alimentée par des éléments tels que l'éclosion tardive de
Molière en tant qu'auteur, son prétendu manque d'éducation et de culture, ainsi
que l'absence de preuves manuscrites permettant de réfuter directement cette
hypothèse.