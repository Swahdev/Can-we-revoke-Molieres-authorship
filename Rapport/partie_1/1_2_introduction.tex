\vspace{\baselineskip}
\hspace{0,5cm}Au début des années 2000, Cyril et Dominique Labbé, deux chercheurs, ont avancé
l'idée selon laquelle Corneille aurait écrit pour Molière. Leur méthode consiste
à mesurer une "distance inter-textuelle" qui évalue la différence de lexique
entre les textes des deux auteurs. Si cette distance ne dépasse pas un certain
seuil, les deux pièces sont considérées comme écrites par le même auteur. Ils se
basent également sur le fait que de nombreux dramaturges de l'époque signaient
leurs œuvres sous le nom de "comédien poète", ce qui permettait aux véritables
auteurs de rester anonymes tout en bénéficiant de la promotion et de la
représentation de leurs pièces par les acteurs.


\hspace{0,5cm}Cependant, cette méthodologie a été contestée par d'autres chercheurs. Certains
ont souligné que l'implémentation de la méthode de Cyril et Dominique Labbé
pourrait "lisser artificiellement les différences entre les auteurs", en
utilisant une distance euclidienne qui accorde trop de poids aux lemmes
fréquents, réduisant ainsi la disparité entre les fréquences observées de
différentes formes. 
\\D'autres approches ont été proposées pour résoudre le problème de l'attribution
des comédies de Molière. Certaines méthodes utilisent une analyse textuelle et
statistique pour comparer les styles d'écriture, tandis que d'autres adoptent
des approches plus qualitatives en examinant les intrigues, la versification et
les sujets choisis dans les pièces.
