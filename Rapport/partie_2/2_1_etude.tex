\vspace{\baselineskip}
\hspace{0,5cm}Il y a trois ensembles d'œuvres que nous appellerons des corpus.  Un corpus
désigne une collection importante et structurée de textes ou de documents
utilisée pour l'analyse linguistique. Ces corpus peuvent généralement comprendre
une variété de textes tels que des livres, des articles :

\hspace{1cm}- Le premier corpus, appelé "corpus exploratoire", est constitué d'un large
échantillon de comédies en vers. Cet échantillon comprend des pièces d'au moins
5000 mots pour les auteurs ayant écrit au moins trois comédies. Il inclut des
pièces de théâtre de 12 auteurs.

\hspace{1cm}- Le deuxième corpus, appelé "corpus final", est construit pour obtenir un
résultat plus lisible et moins biaisé. Pour éviter les biais liés aux
sous-genres, les chercheurs vont exclure les comédies héroïques et les courtes
farces comiques. Afin d'éliminer le bruit ajouté par de nombreux phénomènes qui
ne sont pas liés aux hypothèses présentées précédemment, ils choisissent de se
concentrer uniquement sur cinq auteurs majeurs de l'époque.  Ce corpus final
comprend 37 pièces de T. et P. Corneille, Molière, Rotrou et Scarron.

\hspace{1cm}- Le troisième corpus sert de test pour vérifier la précision de leur approche. Il
est constitué de comédies en vers écrites après la mort de P. Corneille et
Molière.