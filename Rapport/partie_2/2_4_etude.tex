\vspace{\baselineskip}
\hspace{0,5cm}Dans le cadre d'une approche statistique pour l'attribution d'auteurs par le
biais de l'apprentissage automatique, l'algorithme de
clusterisation hiérarchique a été appliqué à chacun des corpus. Cet
algorithme est un type spécifique d'algorithme de regroupement utilisé dans le
domaine de l'apprentissage automatique et de l'analyse de données. Il s'agit
d'une approche ascendante où chaque point de données est initialement considéré
comme un cluster séparé, puis fusionné de manière itérative en fonction de leur
similarité.

\hspace{0,5cm}Un cluster, dans le contexte de l'analyse de regroupement, fait
référence à un groupe de points de données partageant des similitudes ou
présentant des schémas lorsqu'ils sont comparés à d'autres points de données.
Ces points de données sont regroupés en fonction de certains critères tels que
la proximité dans l'espace des caractéristiques ou la similarité dans les
valeurs des attributs. Les clusters sont formés en se basant sur des mesures de
similarité ou de dissimilarité utilisées dans l'algorithme de regroupement.